\documentclass[12pt]{article}
\usepackage{sbc-template}
\usepackage{graphicx,url}
\usepackage[utf8]{inputenc}  
\sloppy
\title{Trabalho sobre Sintaxe e Semântica\\ Linguagems de Programação: Documentação do programa}
\author{Bruno P. Baeta\inst{1}, Davi D. Magalhães\inst{2}, Matheus R. Figueiredo\inst{3} }
\address
{
Instituto de Ciências Exatas e de Informática \\ PONTIFÍCIA UNIVERSIDADE CATÓLICA DE MINAS GERAIS (PUCMG)\\
Belo Horizonte -- MG -- Brazil
}
\begin{document} 
\maketitle




%%%%%%%%%%%%%%%%%%%%%Introdução%%%%%%%%%%%%%%%%%%%%%%%%%%%%%%%%%%%%%%

\begin{resumo} 
Esta documentação tem como objetivo explicar o funcionamento do programa feito para o Trabalho de Sintaxe e Semântica da matéria de Linguagens de Programação do Professor Hugo Bastos de Paula.
  \\\textbf{\keyword{Palavras-chave: }}\LaTeX. Python.
\end{resumo}

%%%%%%%%%%%%%%%%%%%%%Introdução%%%%%%%%%%%%%%%%%%%%%%%%%%%%%%%%%%%%%%


%%%%%%%%%%%%%%%%%%%%%Menu%%%%%%%%%%%%%%%%%%%%%%%%%%%%%%%%%%%%%%

\section{Menu}
O programa começa pelo menu, que está no arquivo "main.py" que, ao ser compilado, apresenta uma simples interface que nos fornece 4 opções: \\
1- \ Função de criar produto (tipos de dados e dicionários) \\
2- \ Função recursiva que apresenta a sequencia de Fibonacci (recursividade e vetores) \\
3- \ Função para cadastrar clientes em um arquivo externo (manipulação de arquivos) \\
4- \ Sair do Sistema \\
\\
Cada opção (com exceção da 4ª) chama uma função diferente que será explicado nos próximos tópicos.\\
O menu exemplifica o uso de estruturas de repetição (while) e condicionais (if), além de entrada (input) e saída de dados (print) em Python.

%%%%%%%%%%%%%%%%%%%%%Menu%%%%%%%%%%%%%%%%%%%%%%%%%%%%%%%%%%%%%%


%%%%%%%%%%%%%%%%%%%%%Função 1%%%%%%%%%%%%%%%%%%%%%%%%%%%

\section{Função 1}
A função 1, definida no código como "func1()", declara variáveis globais que armazenarão dados que serão pedidos ao usuário.\\
Os dados pedidos estão relacionados ao cadastro de um produto. Cada dado pedido é de um tipo diferente. Sendo eles string (nome do produto), int (quantidade desse produto no estoque), float (preço do produto) e boolean (se o produto pode ou não ser consumido por menores de idade). Todos os dados são testado para saber se são do tipo correto.\\
Após a entrada de dados pelo usuário, o valor das variáveis é armazenado em um dicionário chamado "produto", que, então, é impresso na tela.\\
Essa função consegue exemplificar muito bem o uso de tipos de dados e dicionários em Python.


%%%%%%%%%%%%%%%%%%%%%Função 1%%%%%%%%%%%%%%%%%%%%%%%%%%%


%%%%%%%%%%%%%%%%%%%%Função 2%%%%%%%%%%%%%%%%%%%%%%%%%%%%%%%%%%%%%

\section{Função 2}
A função 2, definida no código como "func2()", apresenta a sequência de Fibonacci de forma recursiva. Primeiramente a função pergunta ao usuário quantos termos deverão aparecer na sequência e armazena a resposta em uma variável, e então chama uma função recursiva fibonacci() e passa essa variável por parâmetro. A função fibonacci() armazena os valores da sequência de fibonacci em uma lista global previamente declarada. A cada chamada recursiva da função ela passa a variável reduzida de 1 unidade, quando chegar a 0 as chamadas se encerram e então o vetor global é impresso na tela.\\
Essa função consegue exemplificar muito bem o uso de listas e funções recursivas em Python.
%%%%%%%%%%%%%%%%%%%%Função 2%%%%%%%%%%%%%%%%%%%%%%%%%%%%%%%%%%%%%

%%%%%%%%%%%%%%%%%%%%Biblioteca%%%%%%%%%%%%%%%%%%%%%%%%%%%%%%%%%%%%%
\section{Bibliotecas}
No programa, foi criado uma biblioteca para que fosse armazenado funções úteis para ser utilizado no programa. 
Foram criados dois pacotes dentro da pasta 'lib', sendo eles:

\subsection{Interface}
No pacote de Interface, foi utilizado para armazenar funções para a leitura de números inteiros (leiaInt) para auxiliar em uma das entrada de dados na "func3".
Também foi criado 3 funções para auxiliar na criação do menu do programa, sendo elas: linha, cabeçalho e menu.


\subsection{Arquivo}
No pacote de Arquivo, foi utilizado para funções para serem chamadas no programa principal, para que assim se economizasse linhas e sendo assim mais funcional.
Para isso, foi criado as funções:

\textbullet arquivoExiste: Faz uma consulta para ver se o arquivo existe, tentando verificar se o arquivo, caso retorne o erro FileNotFoundError, ele retornará um boolean.

\textbullet criarArquivo: Ele tenta abrir o arquivo passando o parâmetro com o nome do arquivo e de Write+Text.

\textbullet cadastrar: Ele recebe os parâmetros que são declarados no programa e escreve no arquivo criado.

\textbullet lerArquivo: Ele tenta ler o arquivo com o parâmetro de Read+Text, caso retorne uma exceção ele retorna um erro, caso não, ele para cada linha do arquivo, lê e separa eles.
%%%%%%%%%%%%%%%%%%%%Biblioteca%%%%%%%%%%%%%%%%%%%%%%%%%%%%%%%%%%%%%


%%%%%%%%%%%%%%%%%%%%%Função 3%%%%%%%%%%%%%%%%%%%%%%%%%%%%%%%%%%%%%%%%
\section{Função 3}
A função 3, definida no código como "func3()", são chamados métodos do pacote de arquivos da biblioteca para auxiliar na verificação se o arquivo já existe, caso não, ele criará um. Também é declarado um menu para o usuário escolher entre cadastrar uma pessoa, listar as pessoas cadastradas e uma opção para o usuário voltar para o menu principal do programa.

%%%%%%%%%%%%%%%%%%%%%Função 3%%%%%%%%%%%%%%%%%%%%%%%%%%%%%%%%%%%%%%%%
\end{document}